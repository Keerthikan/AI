\section{Sokoban Solver}
This part of the report will, as suggested by the title, present the sokoban solver.
Initially, a brief overview of the various iterations of the method used to solve the problem will be presented, after which the final solution is shown in more detail.
During the process a few optimizations were done to reduce the computational workload of the solver.

\subsection{The Sokoban Map}
A sokoban map consists of walls, goals, diamonds and the man. 
Each of these are represented using simple symbols and characters.
The man is represented by 'M' or 'm' when standing on a goal. 
A diamond is represented by 'J' or 'j' when standing on a goal.
A goal is represented by 'G' and a wall by 'X'.
When necessary, deadlock situations will be represented by a red square while non-deadlocks will have a green square.

\subsection{The Algorithm and A*}
A*, being a well-established algorithm for pathfinding problems is an obvious candidate for creating the solver.
The solver was developed mostly using \cite{stanford} as reference.
A* is a greedy algorithm that, when given an appropriate, underestimating heuristic, is guaranteed to find the shortest path in terms of whatever cost described by the heuristic.
In order to use the algorithm, it is necessary to define the terms used by the algorithm:

\paragraph{State:}
First, it is necessary to define what constitutes a state. 
Any state should be uniquely distinguishable from any other state and should hold the information necessary to describe the goal condition.
In the case of the sokoban solver, the position of the diamonds and the man on the map and the map itself describes a state.
However, it is unnecessary to include the map in the state representation as this is static.
Theoretically, one could leave out the position of the man as well as this is irrelevant when describing the goal condition.
It was chosen to include the position of the man in the state for ease of access in various parts of the implementation.

\paragraph{Initial State:}
\label{sec:initialstate}
This is the starting point of the algorithm. 
In the case of the competition map, see figure \ref{fig:competition}, the state is defined as follows:

\begin{lstlisting}
	Man: {1,6}
	Diamonds: {3,2;3,3;3,4;4,3}
\end{lstlisting}

\paragraph{Goal State:}
This is the state such that, when reached, the problem is solved. 
Upon each iteration of the algorithm each new state is compared to this one in order to decide whether another iteration should be done.
It is worth noting that the position of the man is disregarded when comparing with the goal state as the man has no impact on the win condition of the game.
Additionally, the order of the diamonds is irrelevant.
For the competition map the goal state is defined as:

\begin{lstlisting}
	Man: {0,0}
	Diamonds: {1,8;2,7;2,8;3,8}
\end{lstlisting}

\paragraph{Heuristic:}
The heuristic is the measure by which the algorithm determines which state to treat next.
It should provide a measure of the approximated distance to the goal condition. 
This is done by applying the wavefront algorithm to the map with goals as the initial nodes.
The resulting wavefront map can be seen in figure \ref{fig:competitionwavefront}.
The heuristic is then just $\sum_{i}$wf(Diamonds[i]) where i goes from zero to the number of diamonds.
Using this method means that the heuristic will reflect the distance from each diamond to the goal closest to the diamond. 
This ensures that the heuristic is underestimated, as is required to guarantee the "shortest path" property of A*.
Initially it was considered to add a penalty to switching diamonds, thereby encouraging the algorithm to finish work on one diamond before starting the next.
This approach was thought to result in a more, subjectively, "natural" looking solution of the map.
It was eventually abandoned due to unnecessity and a desire for simplifying the code base.

\paragraph{Cost:}
This is the exact cost of going from the initial state to some state $S_i$. 
In this case this cost is defined simply as the number of steps the man must take in order to reach $S_i$.\\~\\

The parts described above are connected in the \texttt{solve()} function. 
This function holds the implementation of the A*.
It holds the two sets, \texttt{open} and \texttt{closed}.
\texttt{open} should always be sorted such that the state with the lowest heuristic is the first element.
This makes the \texttt{priority\_queue} a good candidate as the PQ allows for simple sorting of the set.
\texttt{closed} should be searched for every new state to ensure that they have not been attempted previously, before being pushed onto \texttt{open}.
In order to easily search for a state \texttt{map} is used. 
A string is used for hashing. It contains the diamond and man positions and is generated from a state. 
The initial state mentioned previously would produce \texttt{3233344316}.\\
The implementation used does allow for some optimization;
Since the order of the diamonds does not uniquely identify one state, by changing the hashing function such that it compares every diamond position in a state, with every position in the string, the number of states pushed onto \texttt{open} could potentially be reduced.\\
When a state is taken from \texttt{open} all of its potential children are found using \texttt{get\_children()}. 
This function utilizes the wavefront algorithm as well.
By applying a wavefront with the diamonds set as walls and the man as the initiator, every reachable position will have a value $>1$.
Every side of each diamond is then checked, first to see if the man can reach that side, then to see if the opposite side of the diamond is a valid new position for the diamond.
If both return true that state is created and saved.

